\section{}
\begin{center}
\begin{tabular}{rrrrr}
\hline
   Node &   Closeness &   Nearness &   Degree &   Adjacency \\
\hline
      1 &      0.0357 &       4.92 &        1 &     -0.667  \\
      2 &      0.0357 &       4.92 &        1 &     -0.667  \\
      3 &      0.0357 &       4.92 &        1 &     -0.667  \\
      4 &      0.0556 &       7.83 &        5 &      0.515  \\
      5 &      0.0357 &       4.92 &        1 &     -0.667  \\
      6 &      0.0625 &       8.5  &        6 &      0.226  \\
      7 &      0.0417 &       6.17 &        3 &     -0.206  \\
      8 &      0.0435 &       6.67 &        4 &      0.0214 \\
      9 &      0.0435 &       6.33 &        3 &     -0.206  \\
     10 &      0.0455 &       6.83 &        4 &      0.136  \\
     11 &      0.0455 &       6.83 &        4 &     -0.0143 \\
     12 &      0.0312 &       4.5  &        1 &     -0.6    \\
\hline
\end{tabular}
\end{center}

\section{}
\subsection*{London}
\begin{adjustwidth}{-1.3in}{-1.3in}\begin{center}
\begin{tabular}{lrlrlrlr}
\hline
 Node                &   Closeness & Node                &   Near. & Node                &   Deg. & Node                &   Adj. \\
\hline
 greenpark           &    0.000307 & greenpark           &    59.5 & bakerstreet         &      7 & paddington          &  0.48  \\
 westminster         &    0.000299 & bank                &    59.3 & stratford           &      7 & stratford           &  0.475 \\
 bondstreet          &    0.000299 & kingscrossstpancras &    58.7 & kingscrossstpancras &      7 & kingscrossstpancras &  0.46  \\
 kingscrossstpancras &    0.000299 & bakerstreet         &    58.3 & paddington          &      6 & bakerstreet         &  0.453 \\
 oxfordcircus        &    0.000298 & oxfordcircus        &    57.8 & oxfordcircus        &      6 & canningtown         &  0.417 \\
 bank                &    0.000295 & waterloo            &    57.8 & waterloo            &      6 & stockwell           &  0.4   \\
 waterloo            &    0.000295 & bondstreet          &    57.3 & bank                &      6 & chalfont\&latimer    &  0.4   \\
 bakerstreet         &    0.000294 & westminster         &    56.3 & earlscourt          &      6 & blackhorseroad      &  0.4   \\
 euston              &    0.000291 & euston              &    55.7 & westham             &      6 & willesdenjunction   &  0.365 \\
 victoria            &    0.00029  & liverpoolstreet     &    55.2 & greenpark           &      6 & earlscourt          &  0.339 \\
 farringdon          &    0.00029  & shadwell            &    54.3 & canningtown         &      6 & westham             &  0.321 \\
 angel               &    0.00029  & moorgate            &    54.2 & willesdenjunction   &      5 & surreyquays         &  0.317 \\
 hydeparkcorner      &    0.000288 & highbury\&islington  &    53.7 & liverpoolstreet     &      5 & shadwell            &  0.309 \\
 moorgate            &    0.000286 & warrenstreet        &    53.3 & turnhamgreen        &      5 & sydenham            &  0.3   \\
 barbican            &    0.000285 & finchleyroad        &    53.3 & euston              &      5 & finchleycentral     &  0.3   \\
 oldstreet           &    0.000285 & victoria            &    53   & shadwell            &      5 & turnhamgreen        &  0.294 \\
 warrenstreet        &    0.000284 & embankment          &    53   & piccadillycircus    &      4 & waterloo            &  0.289 \\
 liverpoolstreet     &    0.000284 & piccadillycircus    &    52.7 & embankment          &      4 & nottinghillgate     &  0.286 \\
 highbury\&islington  &    0.000284 & tottenhamcourtroad  &    52.4 & shepherdsbush       &      4 & holborn             &  0.25  \\
 piccadillycircus    &    0.000284 & regentspark         &    52.3 & nottinghillgate     &      4 & wembleypark         &  0.25  \\
 eustonsquare        &    0.000284 &                     &         & bondstreet          &      4 & westhampstead       &  0.25  \\
                     &             &                     &         & tottenhamcourtroad  &      4 & finsburypark        &  0.25  \\
                     &             &                     &         & holborn             &      4 &                     &        \\
                     &             &                     &         & mileend             &      4 &                     &        \\
                     &             &                     &         & victoria            &      4 &                     &        \\
                     &             &                     &         & westminster         &      4 &                     &        \\
                     &             &                     &         & moorgate            &      4 &                     &        \\
                     &             &                     &         & actontown           &      4 &                     &        \\
                     &             &                     &         & westbrompton        &      4 &                     &        \\
                     &             &                     &         & whitechapel         &      4 &                     &        \\
                     &             &                     &         & wembleypark         &      4 &                     &        \\
                     &             &                     &         & westhampstead       &      4 &                     &        \\
                     &             &                     &         & finchleyroad        &      4 &                     &        \\
                     &             &                     &         & londonbridge        &      4 &                     &        \\
                     &             &                     &         & canadawater         &      4 &                     &        \\
                     &             &                     &         & canarywharf         &      4 &                     &        \\
                     &             &                     &         & camdentown          &      4 &                     &        \\
                     &             &                     &         & leicestersquare     &      4 &                     &        \\
                     &             &                     &         & stockwell           &      4 &                     &        \\
                     &             &                     &         & finsburypark        &      4 &                     &        \\
                     &             &                     &         & highbury\&islington  &      4 &                     &        \\
                     &             &                     &         & blackhorseroad      &      4 &                     &        \\
                     &             &                     &         & surreyquays         &      4 &                     &        \\
                     &             &                     &         & poplar              &      4 &                     &        \\
\hline
\end{tabular}
\end{center}\end{adjustwidth}
\subsection*{Rodget}
\begin{center}
\begin{tabular}{rrrrrrrr}
\hline
   Node &   Closeness &   Node &   Near. &   Node &   Deg. &   Node &   Adj. \\
\hline
    660 &    0.000496 &    660 &     540 &    660 &    145 &    562 &  0.475 \\
    651 &    0.000491 &    470 &     534 &    674 &    136 &    674 &  0.449 \\
    470 &    0.00049  &    674 &     533 &    470 &    135 &    660 &  0.446 \\
    674 &    0.00049  &    506 &     530 &    506 &    128 &    651 &  0.444 \\
    506 &    0.000489 &    651 &     527 &    562 &    123 &    470 &  0.436 \\
    658 &    0.000489 &    658 &     526 &    697 &    123 &    506 &  0.414 \\
     86 &    0.000488 &    562 &     525 &    557 &    121 &    222 &  0.411 \\
    222 &    0.000487 &    222 &     524 &    698 &    119 &    846 &  0.399 \\
    562 &    0.000484 &    698 &     523 &    619 &    118 &    619 &  0.397 \\
    698 &    0.000484 &    697 &     523 &    651 &    117 &     86 &  0.394 \\
    557 &    0.000482 &    557 &     523 &    658 &    117 &    698 &  0.39  \\
    697 &    0.000481 &    722 &     521 &    222 &    116 &    722 &  0.383 \\
    722 &    0.000481 &     86 &     521 &    722 &    116 &    697 &  0.375 \\
    766 &    0.000481 &    766 &     516 &    167 &    110 &    557 &  0.371 \\
    469 &    0.00048  &    469 &     516 &    486 &    109 &    539 &  0.366 \\
    713 &    0.000478 &    539 &     515 &    846 &    109 &    167 &  0.362 \\
    539 &    0.000477 &    619 &     515 &    539 &    108 &    766 &  0.362 \\
    688 &    0.000477 &    713 &     514 &    469 &    105 &    658 &  0.36  \\
    641 &    0.000475 &    846 &     514 &    507 &    105 &    847 &  0.357 \\
    721 &    0.000475 &    486 &     513 &    713 &    105 &     46 &  0.347 \\
        &             &        &         &    766 &    105 &        &        \\
\hline
\end{tabular}
\end{center}
\subsection*{CCSB Y2H}
\begin{adjustwidth}{-1.2in}{-1.2in}\begin{center}
\begin{tabular}{lrlrlrlr}
\hline
 Node    &   Closeness & Node    &   Near. & Node    &   Deg. & Node    &   Adj. \\
\hline
 YLR291C &    0.000336 & YLR291C &     385 & YLR291C &     86 & YCR106W &  0.892 \\
 YBR261C &    0.000297 & YLR423C &     327 & YLR423C &     58 & YIR038C &  0.876 \\
 YPL070W &    0.000295 & YBR261C &     325 & YIR038C &     51 & YML051W &  0.865 \\
 YCL028W &    0.000292 & YPL070W &     316 & YBR261C &     42 & YLR423C &  0.856 \\
 YPL049C &    0.000289 & YPL049C &     310 & YDR510W &     40 & YLR291C &  0.848 \\
 YLR423C &    0.000289 & YCL028W &     305 & YDR479C &     36 & YDR510W &  0.832 \\
 YNL044W &    0.000283 & YHR113W &     298 & YDR100W &     30 & YDL100C &  0.832 \\
 YHR113W &    0.000282 & YDR510W &     297 & YML051W &     29 & YDR479C &  0.79  \\
 YOR284W &    0.000282 & YNL044W &     296 & YPL094C &     28 & YPL094C &  0.747 \\
 YLR245C &    0.00028  & YOR284W &     294 & YPL049C &     27 & YMR070W &  0.722 \\
 YBR080C &    0.000279 & YKR034W &     294 & YPL070W &     27 & YBR261C &  0.72  \\
 YPL088W &    0.000278 & YLR245C &     291 & YAR027W &     25 & YPL004C &  0.719 \\
 YGR267C &    0.000277 & YDL239C &     291 & YNL189W &     22 & YDR100W &  0.665 \\
 YHL018W &    0.000277 & YGL153W &     290 & YDL100C &     22 & YAR027W &  0.644 \\
 YBR233W &    0.000276 & YPL088W &     289 & YDR448W &     21 & YER125W &  0.629 \\
 YMR095C &    0.000276 & YBR080C &     289 & YCR106W &     20 & YIR033W &  0.625 \\
 YDR256C &    0.000276 & YNL229C &     288 & YKR034W &     19 & YDR448W &  0.617 \\
 YOR095C &    0.000275 & YHL018W &     288 & YIR033W &     17 & YKL117W &  0.615 \\
 YHR112C &    0.000275 & YMR095C &     287 & YML029W &     17 & YML029W &  0.59  \\
 YKR034W &    0.000275 & YOL034W &     287 & YHR113W &     16 & YJL019W &  0.585 \\
\hline
\end{tabular}
\end{center}\end{adjustwidth}

We can dive a little deeper into comparing Nearness Centrality with Closeness Centrality and adjacency centrality with degree centrality by preforming a simple Spearman rank correlation analysis.
We can work out the rank of each node under a particular centrality metric by ordering the nodes by that metric.
Then preform Spearman rank corelation between the rank of nodes under 2 centrality metrics.
The correlation coefficient for Nearness and Closeness centrality, and for Degree and Adjacency Centrality are tabulated below. 
As a reminder, a coefficient of 1 means they are highly positively correlated and a coefficient of 0 means they have no corelation. 
\begin{center}
\begin{tabular}{lrr}
\hline
 Dataset   &   Closness \& Nearness &   Degree \& Adjacency \\
\hline
 London    &              0.985233 &             0.658266 \\
 Roget     &              0.999154 &             0.91161  \\
 CCSB Y2H  &              0.993334 &             0.731789 \\
\hline
\end{tabular}
\end{center}


Using the corelation coefficient and observing the top 20 results for each dataset.
We can say that nearness centrality is a good alternative for closeness centrality.
Also we can say that adjacency centrality can be used as an alternative to degree centrality, although their relationship is less strong than that of nearness and closeness. 

