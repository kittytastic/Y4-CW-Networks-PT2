% TODO - undirected vs directed edges Q2 and Q3
\section{}
If we pre-building routes, the routing of a package is very simple computationally, just requiring a lookup at each node.
However pre-building routes has the overhead of building these routes in the setup stage
This includes calculating routes that may never be taken in practice depending on the network's traffic patten.
There is also the overhead of storing these routes, which may require a large amount of space depending on the complexity of the route and number of destinations.
Pre-built routes also forgo any fault tolerance; that is, unless we introduce additional protocols/mechanism to identify and rebuild broken routes, which introduces further overheads.


On the other hand, building routes on the fly has the major benefit of being far more fault tolerant as routes are created using a much more up-to-date knowledge of the current state of the network.
However, this has the obvious overhead that routes need to be recalculated every time so many calculations may be repeated.

Additionally, pre-building routes offers the opportunity of calculating routes using global knowledge of the network (before any faults happen).
As opposed to the local knowledge building routes on the fly is limited to.
This means that pre-built routes could optimise network performance properties, such as load balancing, better than routes built on the fly with only local knowledge. 

\section{}
\subsection{}
We can calculate the number of paths of each length:
\begin{center}
    \begin{tabular}{c|c}
        Path length & Count\\
        \hline
        0 & ${4 \choose 0} = 1$ \\
        1 &  ${4 \choose 1} = 4$\\
        2 &  ${4 \choose 2} = 6$\\
        3 &  ${4 \choose 3} = 4$\\
        4 &  ${4 \choose 4} = 1$\\
    \end{tabular}
\end{center}
    
Given we need don't need to store the source in a path we can assume that we have the source stored somewhere else and hence don't need an entry for it in the table.
In total we require 

\begin{align*}
    4\sum \text{Path length} \cdot \text{Count}&=\\
    &=4\cdot(1\cdot 4 + 2 \cdot 6 + 3 \cdot 4+ 4 \cdot 1)\\
    &= 128 \texttt{ bits}
\end{align*}
\subsection{}
In the table we will store the destination and next node in the path for each entry in the table.
\[\text{row entries}=2\]
Again we assume we have this nodes label stored elsewhere hence we don't need a row for it.
\[\text{destinations that aren't self}=2^4-1=15\]
Hence we require
\[2\cdot 15 \cdot 4 = 120 \texttt{ bits}\]

\subsection{}
We will adapt the scheme from \textit{b}.
Firstly we will move away from storing the label of the next node and instead store the index of the outgoing channel.
We can define the channels for $Q_4$ as

\[\{(b_0, \ldots, b_{i-1}, 0, b_{i+1}, \ldots, b_{3})\sim (b_0, \ldots, b_{i-1}, 1, b_{i+1}, \ldots, b_{3})\}\]

For every node we then define the channel index  as $i$. 
As $i\leq 3$ we can encode it in $2 \texttt{ bits}$.

Secondly we will do away with the labels of the destination nodes in the table, instead inferring the destination node by using the index of the table row. 
We will continue with the assumption that this node's label $u$ is stored elsewhere and hence can be excluded from the table.
With this in mind we can use the following formula to select the row for a destination node $t$  
\[ \text{row index} = 
    \begin{cases} 
      t & t<u \\
      t-1 & t>u \\
      \text{N/A} & t=u 
   \end{cases}
\]


Hence we will require:
\[1 \texttt{ row entry}\cdot 2 \texttt{ bits}\cdot 15 \texttt{ columns}  = 30 \texttt{ bits}\]


\section{}
\subsection{}
\subsubsection*{Circulant Graph $Cn=(V_n,E_n)$}
\[V_n=\{k: 0\leq k < n\}\]
\begin{align*}
    E_n =\phantom{\cup}&\{(k)\sim(k+1 \mod n): 0 \leq k < n)\}\\
    \cup&\{(k)\sim(k+\frac{n}{2} \mod n): 0 \leq k < \frac{n}{2})\}
\end{align*}
%edge1. $i+1 \mod n$, edge2 $i-1 \mod n$, edge3 $i+\frac{n}{2} \mod n$

\subsubsection*{Petersen Graph $P_n=(V_n,E_n)$}
\[ V_n = \{k: 0\leq k < n\}\]
\begin{align*}
    E_n =\phantom{\cup}&\{(k)\sim (k+2 \mod n): k=2j \setbr k,j \in  \mathbb{Z} \setbr 0\leq k<n\}\\
    \cup&\{(k)\sim(k+1 \mod n): k=2j\setbr k,j \in \mathbb{Z} \setbr  0\leq k<n\}\\ 
    \cup&\{(k)\sim (k+4 \mod n): k=2j+1\setbr k,j \in \mathbb{Z} \setbr  0\leq k<n\} 
\end{align*}
%if $i$ is even  
%edge1. $i+1 \mod n$ edge2. $i+2 \mod n$ edge3. $i-2 \mod n$
%if $i$ is odd
%edge1. $i-1 \mod n$ edge2. $i+4 \mod n$ edge 3. $i-4 \mod n$

\subsection{}
\subsubsection*{Circulant Graph}
This can be accomplished as shown by the following possible routing.

$t=1$
\begin{center}
\begin{tabular}{r|ccccccccc}
 Node  & 1 & 2 & 3 & 4 & 5 & 6 & 7 & 8 & 9\\
 \hline
 Received message for node & 3 &  &  &  & 4 &  &  &  & 7 \\
 Node satisfied   &  &  &  &  &  &  &  &  & 
\end{tabular}
\end{center}

$t=2$
\begin{center}
\begin{tabular}{r|ccccccccc}
 Node  & 1 & 2 & 3 & 4 & 5 & 6 & 7 & 8 & 9\\
 \hline
 Received message for node & 2 & 3 &  & 4 & 6 &  &  & 7 & 8 \\
 Node satisfied   &  &  &  & \checkmark &  &  &  &  & 
\end{tabular}
\end{center}

$t=3$
\begin{center}
\begin{tabular}{r|ccccccccc}
 Node  & 1 & 2 & 3 & 4 & 5 & 6 & 7 & 8 & 9\\
 \hline
 Received message for node & 1 & 2 & 3 &  & 5 & 6 & 7 & 8 & 9 \\
 Node satisfied   &  \checkmark&  \checkmark&  \checkmark& \checkmark &  \checkmark&  \checkmark&\checkmark  &  \checkmark&\checkmark 
\end{tabular}
\end{center}
\subsubsection*{Petersen Graph}
This can be accomplished as shown by the following possible routing.

$t=1$
\begin{center}
\begin{tabular}{r|ccccccccc}
 Node  & 1 & 2 & 3 & 4 & 5 & 6 & 7 & 8 & 9\\
 \hline
 Received message for node & 7 & 4 &  &  &  &  &  & 6 & \\
 Node satisfied   &  &  &  &  &  &  &  &  & 
\end{tabular}
\end{center}

$t=2$
\begin{center}
\begin{tabular}{r|ccccccccc}
 Node  & 1 & 2 & 3 & 4 & 5 & 6 & 7 & 8 & 9\\
 \hline
  Received message for node  & 5 & 3 &  & 4 &  & 6 & 7 & 9 & \\
 Node satisfied   &  & & & \checkmark & & \checkmark & \checkmark &   &   
\end{tabular}
\end{center}

$t=3$
\begin{center}
\begin{tabular}{r|ccccccccc}
 Node  & 1 & 2 & 3 & 4 & 5 & 6 & 7 & 8 & 9\\
 \hline
 Received message for node  & 1 & 2 & 3 &  & 5 &  &  & 8 & 9 \\
 Node satisfied   &  \checkmark& \checkmark& \checkmark& \checkmark & \checkmark& \checkmark & \checkmark &  \checkmark &  \checkmark 
\end{tabular}
\end{center}

\subsection{}
This is possible for the Petersen graph.
This is because it has diameter 2 hence we can reach any node from any other node in 2 hops which in this example corresponds to sending a message size b in 2 seconds.

It is not possible to do for the Circulant graph. Take node 7 and 3, you can see that there are no paths of length 2 from node 0 to them.
If we were to create a breath first search tree starting at node 0 we would find them at depth 3.
In this broadcast scheme we can descend 1 layer of the BFS tree per time-step, hence it is impossible to reach nodes 7 and 3 at depth 3 within 2 time-steps. 

\subsection{}
\subsubsection*{Circulant Graph}
Single-node broadcast from node 0:
\begin{center}
\begin{tabular}{r|cccccccccc}
 Node  &0& 1 & 2 & 3 & 4 & 5 & 6 & 7 & 8 & 9\\
 \hline
 Path  & - & 0,1 & 0,1,2 & 0,5,4,3 & 0,5,4 & 0,5 & 0,5,6 & 0,5,6,7 & 0,9,8 & 0,9 \\
\end{tabular}
\end{center}

We observe in this single-node broadcast that the load on the internal channel $0\sim 5$ is 5.
Also the load on the external channels $(0\sim 1, 1\sim 2, \ldots)$ is $(2,1,0,1,2,2,1,0,1,2)$.
By applying this single-node broadcast patten to every node we can obtain a routing for total exchange.

We will calculate the load on every chanel by using a simple geometric automorphism.
This automorphism will be a rotation of the graph $\frac{1}{10}$ of a cycle.
Applied to the broadcast scheme this automorphism can map the start node to any other node.
We can observe that this automorphism does not change the type of a chanel.
That is to say that, internal channels remain internal and external channels remain external.

Using this observation we can see that all the internal channels will have a load of 10 as a load of 5 will be applied to each channel from 2 possible transforms of the broadcast scheme.
For the external chanel we can explore what happens at some fixed chanel $x_0$.
Next we will label the remaining channels $x_1, x_2, \ldots x_9$ counterclockwise from $x_0$.
For every clockwise rotation of the broadcast patten we can see that $x_0$ will have the load of $x_i$ where $i$ is the number of clockwise rotations preformed.
As we preform all 10 possible clockwise rotations $x_0$ will have the total load of $x_0 + x_1 \ldots x_9$.
As every channel external could be $x_0$ we know that every external channel must have the same load.
Therefore, we calculate the load on every external channel to be $2+1+0+1+2+2+1+0+1+2=12$.

In summary the maximal load of $12$ is on the external channels $i\sim i+1 (\mod 10)$ and the 5 internal channels $i\sim i+5 (\mod 10)$ have load $10$.

\subsubsection*{Petersen Graph}

Single-node broadcast from node 0:
\begin{center}
\begin{tabular}{r|cccccccccc}
 Node  &0& 1 & 2 & 3 & 4 & 5 & 6 & 7 & 8 & 9\\
 \hline
 Path  & - & 0,1 & 0,2 & 0, 2, 3 & 0,2,4 & 0,1,5 & 0,8,6 & 0,1,7 & 0,8 & 0,8,9 \\
\end{tabular}
\end{center}

We will follow a similar approach as as we did in the Circulant Graph, approaching the problem geometrically.
We will start by grouping our undirected channels into 3 sets: 
\begin{itemize}
    \item External channels: $(0,2)$, $(2,4)$, $(4,6)$, $(6,8)$, $(8,0)$ 
    \item Cross channels: $(0,1)$, $(2,3)$, $(4,5)$, $(6,7)$, $(8,9)$ 
    \item Internal channels: $(1,5)$, $(1,7)$, $(3,7)$, $(3,9)$, $(5,9)$
\end{itemize}
Next we will define our 2 automorphism: \texttt{rotate} - a simple $\frac{1}{5}$ rotation clockwise; and \texttt{inside-out} a automorphism that maintains the structure of the graph but external channels and internal channels are switched and cross channels are maintained.
\texttt{inside-out} can be defined by the following node mapping: 
\begin{center}
\begin{tabular}{r|cccccccccc}
 Node  &0& 1 & 2 & 3 & 4 & 5 & 6 & 7 & 8 & 9\\
 \hline
 Map to node  & 1 & 0 & 7 & 6 & 3 & 2 & 9 & 8 & 5 & 4 \\
\end{tabular}
\end{center}
and the new external and internal sets can be defined similarly on the mapping using fig 1 as a geometric reference.
It is an important observation that channels in the internal set are now in the external set and visa versa, while the cross channel set remains unchanged.

We will again use a similar argument to that in the Circulant graph that, by virtue of applying \texttt{rotate} to our broadcast scheme, after 5 rotations every channel in the External set will have an equivalent load - which is the sum of the load of every channel in that set.
Likewise the same holds true for the Cross set and Internal set.
Therefore after using 5 \texttt{rotate} transforms to the broadcast scheme we have loads:
\begin{itemize}
    \item External channels: $3+1+1+3=8$ 
    \item Cross channels: $1+3+1=5$ 
    \item Internal channels: $1+1=2$
\end{itemize}
Now we need to address how to transform the broadcast scheme so that internal node can use it.
For this use can apply the \texttt{inside-out} transform followed by a series of \texttt{rotate} transforms.
As we know that the the \texttt{inside-out} transform switches the external and internal set; we can, by symmetry, calculate the load for each set after we preform all 5 possible \texttt{rotates}:
\begin{itemize}
    \item External channels:  $1+1=2$
    \item Cross channels: $1+3+1=5$ 
    \item Internal channels: $3+1+1+3=8$
\end{itemize}

To apply our broadcast scheme to every node in the graph (creating our routing) we simply need to combine the previous 2 procedures, therefore giving us a total load for every channel in each set of:
\begin{itemize}
    \item External channels:  $8+2=10$
    \item Cross channels: $5+5=10$ 
    \item Internal channels: $2+8=10$
\end{itemize}

In summary our routing for total exchange in the Petersen graph applies a load of 10 on every channel in the graph, hence 10 is the maximal load. 